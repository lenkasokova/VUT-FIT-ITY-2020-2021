\documentclass[11pt, twocolumn, a4paper]{article}

\usepackage[utf8x]{inputenc}
\usepackage[left=1.5cm,text={18cm, 25cm},top=2.5cm]{geometry}
\usepackage[IL2]{fontenc}
\usepackage[czech]{babel}
\usepackage{times}
\usepackage{verbatim}
\usepackage{enumitem}

\usepackage{amsmath}
\usepackage{amsthm}
\usepackage{amsfonts}

\newtheorem{definice}{Definice}
\newtheorem{veta}{Věta}

\renewcommand\labelitemi{$\bullet$}

\begin{document}
\begin{titlepage}
\begin{center}
    \Huge
    \textsc{Fakulta informačních technologií\\
    Vysoké učení technické v~Brně}\\
    
    \vspace{\stretch{0.382}}
    
    \LARGE Typografie a publikování\,--\,2. projekt\\
    Sazba dokumentů a matematických výrazů
    
    \vspace{\stretch{0.618}}
\end{center}
{\Large
2021
\hfill Lenka Šoková (xsokov01)}
\end{titlepage}

\section*{Úvod}
\setlength{\parindent}{11pt}
V~této úloze si vyzkoušíme sazbu titulní strany, matematických vzorců, prostředí a~dalších textových struktur obvyklých pro technicky zaměřené texty (například rovnice~\eqref{eq:first} nebo Definice~\ref{def:def_1} na straně~\pageref{def:def_1}). Rovněž si vyzkoušíme používání odkazů \verb|\ref| a~\verb|\pageref|.

Na titulní straně je využito sázení nadpisu podle optického středu s~využitím zlatého řezu. Tento postup byl
probírán na přednášce. Dále je použito odřádkování se
zadanou relativní velikostí 0.4\,em a~0.3\,em.

V~případě, že budete potřebovat vyjádřit matematickou
konstrukci nebo symbol a~nebude se Vám dařit jej nalézt
v samotném \LaTeX u, doporučuji prostudovat možnosti balíku maker \AmS-\LaTeX.

\section{Matematický text}
\setlength{\parindent}{11pt}
Nejprve se podíváme na sázení matematických symbolů
a~výrazů v~plynulém textu včetně sazby definic a~vět s~využitím balíku \verb|amsthm|. Rovněž použijeme poznámku pod
čarou s~použitím příkazu \verb|\footnote|. Někdy je vhodné
použít konstrukci \verb|\mbox{}|, která říká, že text nemá být
zalomen.

\begin{definice} \label{def:def_1}
\emph{Rozšířený zásobníkový automat} (RZA) je definován jako sedmice tvaru $A = (Q, \Sigma, \Gamma, \delta , q_0, Z_0, F)$, kde:

\begin{itemize}[leftmargin=0.8cm]
    \item $Q$ je konečná množina \emph{vnitřních (řídicích) stavů},
    
    \item $\Sigma$ je konečná \emph{vstupní abeceda},
    
    \item $\Gamma$ je konečná \emph{zásobníková abeceda},
    
    \item$\delta$ je \emph{přechodová funkce }$Q\times(\Sigma\cup\{\epsilon\})\times\Gamma^*\rightarrow{}2^{Q\times\Gamma^*}$,
    
    \item $q_0\in Q$ je \emph{počáteční stav}, $Z_0\in\Gamma$ je \emph{startovací symbol zásobníku a} $F\subseteq Q$ je množina \emph{koncových stavů}.

\end{itemize}
\end{definice}
\setlength{\parindent}{11pt}
Nechť $P = (Q, \Sigma, \Gamma, \delta, q_0, Z_0, F)$ je rozšířený zásobníkový automat. \emph{Konfigurací} nazveme trojici $(q, w, \alpha)\in Q\times\Sigma^*\times\Gamma^*$, kde $q$ je aktuální stav vnitřního řízení, $w$ je dosud nezpracovaná část vstupního řetězce a~$\alpha = Z_{i_1}Z_{i_2}\ldots Z_{i_k}$ je obsah zásobníku\footnote{$Z_{i_1}$ je vrchol zásobníku}.

\subsection{Podsekce obsahující větu a odkaz} \begin{definice} \label{def:def_2} \emph{Řetězec} $w$ \emph{nad abecedou $\Sigma$ je přijat RZA} $A$ jestliže $(q_0, w, Z_0)\underset{A}{\overset{*}{\vdash}}(q_F, \epsilon, \gamma)$ pro nějaké $\gamma\in\Gamma^*$ a $q_F\!\in\!F$. Množinu $L(A) = \{w\mid w\text{ je přijat RZA } A \text{\}} \subseteq \Sigma^*$~nazývame \emph{jazyk přijímaný RZA} $A$.
\end{definice}

\setlength{\parindent}{11pt}
Nyní si vyzkoušíme sazbu vět a důkazů opět s použitím
balíku \verb|amsthm|.

\begin{veta}
Třída jazyků, které jsou přijímány ZA, odpovídá
\emph{bezkontextovým jazykům}.
\end{veta}

\begin{proof} 
V~důkaze vyjdeme z~Definice \ref{def:def_1} a \ref{def:def_2}.
\end{proof}

\section{Rovnice a odkazy}
Složitější matematické formulace sázíme mimo plynulý
text. Lze umístit několik výrazů na jeden řádek, ale pak je
třeba tyto vhodně oddělit, například příkazem \verb|\quad|.

$$\sqrt[i]{x^3_i} \quad\text{kde } x_i \text{ je } i \text{-té sudé číslo splňující}\quad x_i^{x_i^{i^2} + 2}\leq y_i^{x_i^{4}}$$

V rovnici \eqref{eq:first} jsou využity tři typy závorek s~různou
explicitně definovanou velikostí.

\begin{eqnarray}
\label{eq:first}
x & = &\bigg[\Big\{\big[a + b\big]*c\Big\}^d\oplus2\bigg]^{3/2}\\
y & = &\lim_{x\to\infty}\frac{\frac{1}{\log_{10}x}}{\sin^2x + \cos^2x} \nonumber
\end{eqnarray}

V~této větě vidíme, jak vypadá implicitní vysázení limity $\lim_{n\to\infty} f(n)$ v~normálním odstavci textu. Podobně
je to i s~dalšími symboly jako $\prod_{i=1}^{n} 2^{i}$ či $\bigcap_{A\in\mathcal{B}} A$. V~případě vzorců $\lim\limits_{n\to\infty} f(n)$ a~$\prod\limits _{i=1}^n 2^i$ 
jsme si vynutili méně
úspornou sazbu příkazem \verb|\limits|.

\begin{eqnarray}
\int_b^a g(x)\,\mathrm{d}x &=& - \int\limits_a^b f(x)\,\mathrm{d}x
\end{eqnarray}

\section{Matice}
Pro sázení matic se velmi často používá prostředí \verb array  a závorky (\verb|\left|, \verb|\right|).

$$\left( \begin{array}{ccc}
a - b & {\widehat {\xi + \omega}} & \pi\\
\vec {\mathbf{a}} & \overleftrightarrow{AC} & \hat{\beta}
\end{array} \right)
= 1\Longleftrightarrow\mathcal{Q} = \mathbb{R}
$$

$$
\mathbf{A} = \left\| \begin{array}{cccc}
a_{11} & a_{12} & \ldots & a_{1n} \\
a_{21} & a_{22} & \ldots & a_{2n} \\
\vdots & \vdots & \ddots & \vdots \\
a_{m1} & a_{m2} & \ldots & a_{mn} \end{array} \right\|
= 
\left| \begin{array}{ll}
\:t & \:u\: \\
v & \:w
\end{array} \right|
= tw\!-\!uv
$$

Prostředí \verb|array| lze úspěšně využít i jinde.

$$
\begin{binom}
n
k
\end{binom}
=
\left\{
\begin{array}{cl}
0 & \text{pro } k\ < 0 \text{ nebo } k\ > n\\
\frac{n!}{k!(n-k)!}& \text{pro } 0\leq k\ \leq n.
\end{array} \right.
$$

\end{document}