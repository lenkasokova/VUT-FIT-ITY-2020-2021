\documentclass[11pt, a4paper]{article}

\usepackage[left=2cm,text={17cm, 24cm},top=3cm]{geometry}
\usepackage[utf8]{inputenc}
\usepackage[slovak]{babel}
\usepackage[unicode]{hyperref}
\usepackage{textcomp}
\usepackage{times}

\hypersetup{colorlinks = true, hypertexnames = false}


\begin{document}

\begin{titlepage}
\begin{center}
\Huge
\textsc {Vysoké učení technické v~Brně}\\
\huge
\textsc {Fakulta informačních technologií}\\
\vspace{\stretch{0.382}}
    \LARGE {Typografie a~publikování\,--\,4. projekt}\\
    \Huge {Bibliografické odkazy}
\vspace{\stretch{0.618}}
\end{center}

{\Large
\today
\hfill Lenka Šoková }
\end{titlepage}


\section{Úvod}
Typografia je zjednodušene, umenie usporiadať písmená do textu tak, aby výsledný text bol prehľadný, jasný a~vizuálne príťažlivý pre čitateľa. Môže byť oveľa dôležitejšia, ako by sa zdalo. Jej štýľ udáva celkový podtón textu a jeho obsahu, čím oživuje text. Upozorňuje na to, čo by malo byť zdôraznené a zbytočne nezdôrazňuje to, čo je nedôležité. Celkovo udáva vizuálnu hierarchiu dôležitosti informáciam v texte {\cite{Hannah2020}}.

Použitie typografie a správneho fontu písmen je taktiež dôležité aj pri výučbe detí so špeciálnymi potrebami. Vďaka dobrému výberu štýľu textu sa môže pre nich zjednodušiť rozpoznávanie jednotlivých písmen, čo im môže pomôcť pri následnom čítaní a písaní {\cite{Beard2019}}.


\section{Vývoj typografie}
Vznik typografie sa datuje okolo roku 1450 pri vynájdení kníhtlače Johannesom Gutenbergom, no v určitých formách sa objavila už aj predtým a to näjma vo forme Kaligrafie.

\subsection{Kaligrafia a jej rozmach na Blízkom Východe}
Pôvod Kaligrafie siaha až do Starovekého Grécka a neskôr sa rozšírila aj do Blízkeho a Ďalekého východu {\cite{Clayton}}. Na Blízkom Východe sa začala ranná forma kaligrafie používať už pri vzniku islamu a následnom prepisovaní Koránu. V 3. až 4.storočí sa písanie arabského jazyka rozšírilo aj do Egypta, severnej Afriky a Perzie, čo spôsobilo vytvorenie nových štýlov písma. Toto viedlo k zastaveniu vzniku nových štýlov a určeniu jasných pravidiel kaligrafie {\cite{Yusofi1990}}. Napriek jej skorému rozmachu v arbaskom svete, rozvoj typografie a tlačenie arabského písma prišlo vďaka politeckého, náboženského ale aj praktického zámeru až v 18. storočí {\cite{Kampman2011}}.

\subsection{Vznik typového dizajnu}
Vznik Gutenbergovej kníhtlače otvorili dvere aj k zrodeniu typového dizajnu, ktorý odzrkadoval tiež trendy a požiadavky určitých
obdobií v rôznych regiónoch alebo odvetviach. Napríklad románsky typ písma, vznikol pri rozšírení kníhtlače do Itálie, kurzíva vznikla na návrh Aldus Manutius, aby znížil náklady na tlač a pre reklamné potreby bol taktiež vystvorený font na písanie veľkých nadpisov a oznamov (v angličtine display font) {\cite{Ou2019}}.

\section{Písmo}
Písmo v typografií môže mať rôzne štýly - fonty a každý jeden z nich má svoje špecifické určenie. Napríklad pri tvorbe knihy, časopisu, či novín skôr preferujeme pätkové písmo, ktoré vďaka pätiek pomáha čitateľovy udržať sa na riadku. Na zvíraznnenie textu používame opäť šikmé písmo(kurzívu) {\cite{Urbanova2019}}. Každý jeden by mal byť starostlivo zvážený a prispôsobený obsahu textu, čo môže priniesť kvalitný vizuálny výsledok práce {\cite{Cullen2012}}. Taktiež sa treba zamyslieť záležať na vybranie vhodnej veľkosti textu. Napríklad pri písaní nadpisov text nemusí byť veľký, ako samotný odkaz, ale stačí ak je len veľký na toľko, aby upozornilo na zmenu odkazu {\cite{Jelinek1985}}. 


\section{\LaTeX}
Systém \TeX bol vytvorený Donaldom E. Knuthom, aby svoje texty mohol publikovať v požadovovanom tvare, pretože pri tlačení matematických vzorcov dochádzalo k mnohým chybám. Vďaka jeho obecnosti a premyslenosti, od roku 1983 nedošlo k žiadnym zásadním zmenám, no ku ceste zjednošeniu a prirodzenejšiemu sádzaniu textu vznikla nadstavba \LaTeX vytvorená Leslie Lamportom {\cite{Rybicka2003}}.


\bibliographystyle{czechiso}
\renewcommand{\ref name}{Literatúra}
\bibliography{proj4.bib}

\end{document}
